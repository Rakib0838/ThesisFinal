\documentclass
  [
  paper=a4,             % Papiergr��e
  fontsize=11pt,        % Basisschriftgr��e in Punkt
  DIV=calc,             % Autom. Berechnung eines DIV-Werts / manuell: Bereich 6 (gro�e)...15 (kleine R�nder)
  %twoside=true,         % ein-/zweiseitiges Layout (false, true)
  open=right,           % Kapitel nur auf rechter Seite beginnen (bei twoside=true)
  parskip=false,        % Abs�tze mit Zwischenraum (half, auch full = gr��erer Abstand)
  listof=numbered,
  bibliography=totocnumbered,
  %fleqn,               % Formeln linksb�ndig setzen
  %BCOR4mm,             % Bindekorrektur (zus�tzl. innerer Rand)
  ]
  {scrreprt}            % Dokumentklasse

\usepackage[english,ngerman]{babel}         % Neue deutsche Silbentrennung / Sprachpaket
\usepackage[latin1]{inputenc}               % deutsche Tastatureingabe
\usepackage[headsepline,footsepline]{scrpage2}  % Kopf- und Fu�zeilen mit Linien

\usepackage{graphicx,float}					% Einbinden von Grafiken und Bildern
\usepackage{amsmath, amssymb, amstext, amsfonts}	% Mathematikpakete
\usepackage{units}							% Angabe von Einheiten mit $\unit[220]{kV}$

\usepackage{booktabs}						% erweiterte Befehle f�r Tabellen
\usepackage{tabularx}						% Tabellen gleicher Gesamtbreite anlegen (Spaltenbreite kann variabel sein)
\usepackage{multirow}						% Mehrere Zellen miteinander verbinden
\usepackage{colortbl}						% Zeilen oder Zellen in Tabellen f�rben

\usepackage{xspace}                         % Zur Definition von Befehlen f�r mehrgliedrige Abk�rzungen ben�tigt
\usepackage{setspace}						% Seitenr�nder? und Zeilenabstand �nderbar
  %\onehalfspacing							% 1,5-facher Zeilenabstand
\usepackage{color}							% Farben festlegen
\usepackage[absolute]{textpos}              % Absolute Positionierung erm�glichen f�r Titelseite
\usepackage{ifthen}

\usepackage[fixlanguage]{babelbib}			% Zitieren (Quellen referenzieren) / Literaturverzeichnis
  \bibliographystyle{babplain}              % Zitierstil (babplain, babplai3, babalpha, babunsrt, bababbrv, bababbr3)

\usepackage{pdfpages}                       % pdf-Dokumente (teilweise) einbinden

\def\langencond{english}
\def\langdecond{ngerman}
\def\pglay{twoside}
% EOF 